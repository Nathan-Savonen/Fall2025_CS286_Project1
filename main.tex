\documentclass[12pt,a4paper]{cibb}

% Fallback definition for missing \@ordinalM
\makeatletter
\providecommand{\@ordinalM}[2]{#1}
\makeatother

\usepackage{subfigure,graphicx}
\usepackage{amsmath,amsfonts,latexsym,amssymb,euscript,xr}
\usepackage{booktabs}
\usepackage[nodayofweek]{datetime}
\usepackage{hyperref}
\usepackage{fmtcount}
\usepackage[english]{datenumber}
\usepackage[absolute]{textpos}



\usepackage[table]{xcolor}
\usepackage{color,colortbl,tabularx}

\usepackage[english]{babel}
\usepackage[protrusion=true,expansion=true]{microtype}
\usepackage{amsmath,amsfonts,amsthm}
% \usepackage[pdftex]{graphicx}
\usepackage{pifont}

\def\red{\color{red}}
\def\black{\color{black}}
\def\blue{\color{blue}}
\def\magenta{\color{magenta}}



\definecolor{LightBlue}{rgb}{0.88,0.9,0.9}

\newcommand\blfootnote[1]{%
  \begingroup
  \renewcommand\thefootnote{}\footnote{#1}%
  \addtocounter{footnote}{-1}%
  \endgroup
}

\title{\Large $\ $\\ \bf Title for the report}

\author{\large First Member$^1$, Second Member$^{2}$, and Third Member$^{*,3,4}$}
\address{\footnotesize $\ $\\$^1$ First author's department, institute,
city, country. \\
%
$^2$ Second author's department, institute,
city, country. \\
%
$^3$ Third author's primary affiliation department, institute,
city, country.  \\
%
\bigskip

}


\abstract{\small keyword1, keyword2, keyword3, keyword4, keyword5. \normalsize
\\[17pt]
{\bf Abstract.} In this document there are the
main guidelines for preparing your  contribution for the electronic
proceedings of  meeting.
This document will help you to produce the PDF file of your paper.
The paper length must be from 4 to 6 pages.
The language used should consistently follow one of the English variants (Canadian English, British English, US American English, Australian English, etc.).
}

\begin{document}

\renewcommand{\thefootnote}{}
\footnotetext{\small{Article version: \datedate $\;$ h\currenttime  $\;$ CET}}


\thispagestyle{myheadings}
\pagestyle{myheadings}
\markright{\tt Proceedings of CS 286, Fall 2026}%check year


\section{Dataset Exploration}
\label{sec:Dataset Exploration }


\section{Annotation and Signal Exploration}
\label{sec:Annotation}

 \section{Signal Preprocessing}
\label{sec:Signal}

\section{Windowing Strategies}
\label{sec:Windowing}

\section{Feature Extraction}
\label{sec:Feature}

\section{Modeling}
\label{sec:Modeling}

\section{Results}
\label{sec:RESULTS}

The paper  must be formatted  single column, 12 pt,  on standard A4 paper, and its side edges should be 2.5 cm above, down, left, and right, as shown by this document. The maximum length of  the paper is 6 pages.

The references   should   be   cited   in   this   way~\cite{domingos2012few,powell2020tried},   or also~\cite{vcuklina2020review,hansen2014scientific,WWW-GO}.

Software code commands should be written this way: 
\newline
\texttt{x <- 0.5} \newline
\texttt{y <- 0.8} \newline
\texttt{z <- x + y} \newline
\texttt{print(z)} \newline


\subsection{\bf \it Tables and Figures}
\label{sec:TABLES-AND-FIGURES}


Tables and  figures must be placed  in the paper close  where they are
cited.  The caption heading for a table should be placed at the top of
the table,  as shown in~\autoref{tab:RESULTS}.  The caption heading  for a
figure   should   be   placed   below   the  figure,   as   shown   in~\autoref{fig:LOGO}.

\begin{table}[httb!] \small
\centering
    \begin{tabularx}{0.35\textwidth}{ l  l  c }
        \toprule
          \textbf{method} & \textbf{result} & \textbf{interpretation} \\
        \midrule
        \rowcolor{LightBlue} aaa & 0.7 & ccc \\
         ddd & 0.2 & zzz \\
        \rowcolor{LightBlue}  jjj & 0.4 & ppp \\\
         ggg & 0.5 & qqq \\
        \bottomrule
    \end{tabularx}
    \caption{\textbf{Recap of the results}.
    Example table containing results.
    Please alternate the background colors for each row, to facilitate a clearer interpretation\label{tab:RESULTS}}
\end{table}





\subsection{Equations}
\label{sec:EQUATIONS}


Equations should be centered and numbered consecutively, in this way:

\begin{equation}
{\bf y}_{j} = \frac {\sum_{k=1}^{n} (u_{jk})^{m}{\bf x}_{k} }{
\sum_{k=1}^{n} ({ u}_{jk})^{m} } \, \, \ \mbox{\hspace{.5cm}  } \forall j,
\label{eq:FCM_YJ}
\end{equation}
and

\begin{equation}
u_{jk} =
\left\{
\begin{array}{clll}
\left( \sum_{l=1}^{c} \left( \frac {E_{j}({\bf x}_{k})}{E_{l}({\bf x}_{k})}
\right)^\frac{2}{m-1} \right)^{-1} & if & E_{j}({\bf x}_{k})>0 & \forall j,k \\
1 & if & E_{j}({\bf x}_{k})=0 & (u_{lk}=0 \ \ \forall l \neq j) \\
\end{array}
\right.
\label{eq:FCM_UJK}
\end{equation}
and referred as: ~\autoref{eq:FCM_YJ} and ~\autoref{eq:FCM_UJK}.

\begin{figure}[h]
\vspace{3mm}
 \begin{center}
 \includegraphics[width=0.9\textwidth]{./images/example_plot3.pdf}
\caption{\textbf{Example figure}.
Please generate your figure images as PDF files and include them here in the article within the text.
We suggest you to generate your figures with the \texttt{ggplot2} package in \texttt{R} or with \texttt{Matplotlib} in \texttt{Python}.
Regarding colors, please make sure you use a colorblind palette.
This image was generated with \texttt{ggplot2} and then edited with LibreOffice Draw.
\label{fig:LOGO}}
 \end{center}
\vspace{-8mm}
\end{figure}

\section{Conclusion}
\label{sec:CONCLUSIONS}


In this section the authors write the conclusion of the paper. Since this is a short paper, please do not include more than 10 references in the bibliography.


\section*{Conflict of interests}
\label{sec:CONFLICT-OF-INTERESTS}
The authors should declare here any potential conflicts of interests.

\section*{Acknowledgments (optional)}
\label{sec:ACKNOWLEDGMENTS}
The authors would like to thank XXX and YYY for their helpful feedback.

\section*{Funding (optional)}
\label{sec:FUNDING}
This work was supported by the XXX agency (grant number: XXX).

\section*{Availability of data and software code (optional and strongly suggested)}
\label{sec:AVAILABILITY}
Our software code is available at the following URL: XXX.
\newline
\indent Our dataset is available at the following URL: XXX.


\footnotesize
\bibliographystyle{unsrt}
\bibliography{bibliography.bib} 
\normalsize

\end{document}
